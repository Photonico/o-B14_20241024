% abstract for conference

\documentclass[12pt]{article}
\usepackage[utf8]{inputenc}
\usepackage{amsmath} 
\usepackage{authblk}
\usepackage{caption}
\usepackage{geometry}
\usepackage{graphicx}
\usepackage{mhchem}
\usepackage{subcaption}
\geometry{a4paper, margin=40px}

\title{Electronic and Optical Properties of Newly Predicted, Metallic Penta-Bipyramid Boron}
\author[1]{Lu Niu}
\author[1]{Oliver J. Conquest}
\author[1,2]{Carla Verdi}
\author[1*]{Catherine Stampfl}
\affil[1]{School of Physics, The University of Sydney, Sydney 2006, NSW, Australia}
\affil[2]{School of Mathematics and Physics, The University of Queensland, Brisbane, Queensland 4072, Australia}

\date{\today}

\begin{document}
\maketitle

\par Bulk orthorhombic boron (\ce{o-B14}) is a fascinating, recently predicted, three-dimensional boron allotrope~\cite{bulk_boron}. 
Its structure is characterized by edge-sharing pentagonal bipyramids, with diverse bond lengths ranging from $1.644$ to $1.966$ \AA\ as shown in the Figure below.
These structural motifs facilitate the formation of unusual seven-centre-two-electron $\pi$-bonds, which are identified for the first time in three-dimensional boron allotropes.
The presence of these bonds highlights the complexity and versatility of boron chemistry under ambient conditions.
Moreover, this material is predicted to be a superconductor with a $T_c$ value of 29.1 K.

\par In the present work, we explore the atomic and electronic properties of \ce{o-B14} using first-principles calculations based on density-functional theory. Our results provide insight into the electronic band structure, density of states, and bonding characteristics. We also investigate the optical and plasmonic properties of \ce{o-B14}, which reveal distinctive anisotropic behaviors, broadening its potential applications in plasmonics and beyond.

\par \ce{o-B14} exhibits metallic behavior, with three bands crossing the Fermi level.
These bands include steep bands promoting high electron velocities and flat bands near the Fermi level, enhancing the density of states and favoring electronic interactions. Such electronic features suggest the potential for unique transport and quantum properties.
Furthermore, strong orbital hybridisations are observed between the B $2p$ orbitals within the structural units, reinforcing the stability of the bonding framework. The anticipated plasmonic and optical responses, driven by the anisotropic bonding network, offer promising opportunities for advanced optoelectronic and quantum devices.

\begin{figure}[h!]
    \centering
    \includegraphics[width=0.50\textwidth]{../../0.0_structure/structure_b_mac.png}\hfill
    \includegraphics[width=0.50\textwidth]{../../0.0_structure/structure_c_mac.png}
    \caption{Top view (left) and side view (right) of \ce{o-B14} pentagonal bipyramid bulk.}
    \label{fig:o-b14_views}
\end{figure}

\begin{thebibliography}{01}
    \bibitem{bulk_boron} S. Han, Y. Liu, C. Wang, W. Yi, X. Chen, Y. Zhang, and X. Liu, 
    \textit{Superconducting boron allotrope featuring pentagonal bipyramid at ambient pressure}, Phys. Chem. Chem. Phys., \textbf{25}, 15400 (2023).
\end{thebibliography}

\end{document}
